\documentclass{summary}
% \title{Template}
% \begin{document}
% \maketitle
\PaperTitle{Never Split the Difference} % Article title

\Abstract{Lorem ipsum dolor sit amet, consectetuer adipiscing elit. Ut purus elit, vestibulum ut, placerat ac, adipiscing vitae, felis. Curabitur dictum gravida mauris. Nam arcu libero, nonummy eget, consectetuer id, vulputate a, magna. Donec vehicula augue eu neque. Pellentesque habitant morbi tristique senectus et netus et malesuada fames ac turpis egestas. Mauris ut leo. asdfasdf}

%----------------------------------------------------------------------------------------

\begin{document}

\maketitle % Output the title and abstract box
\tableofcontents % Output the contents section
\thispagestyle{empty} % Removes page numbering from the first page

\section{The New Rules}
\subsection{Life is Negotiation}
Negotiation is everywhere and is nothing more than communication with results. First step in negotiation mastery is to get over aversion to it as it is how the world works.

\section{Be a Mirror}

\subsection{Assumptions Blind, Hypotheses Guide}
Engage negotiation with goal of extracting as much information as possible, formulate hypotheses to what they want and test them.

\subsection{Calm the Schizophrenic}
It is very hard to listen well. Our minds engage in selective listening. Do not approach a negotiation being preoccupied with arguments for your side that you are unable to listen attentively. Instead of doing any thinking to what your arguments are, make sole and all-encompassing focus on other person and what they have to say.

Next is to figure out what counterpart actually need (monetairly, emotionally, etc). Can only do this once they are feel save enough to talk about it. First step is actively listen. Second is get down to true motives.

\subsection{Slow. It. Down.}
Going too fast is a common mistake that leaves couterpart feel like they're not being heard. Slow down leads to calm down and builds rapport and trust.

\subsection{The Voice}
Three types of voices: late-night FM DJ, postive/playful, diect/assertive. Most of the time use positive/playful voice. Key is to relax and smile when talking, even if on the phone -- counterpart will pick up on that and become more positive and likely to collaborate. Use late-night FM DJ voice by inflecting voice downward and talking slowly and clearly. It increases authority without making counterpart defensive.

\subsection{Mirroring}
Mirroring body language, tone, etc happens naturally when people trust eachother. For negotiation, mirror focuses on only the words: repeat the last three words or just most critical one to three words. Leads to counterpart elaborating, connecting, and reflecting.

\subsection{How to confront - and get your way - without confrontation}
To repond to aggressive/intimidating authority:
\begin{enumerate}
  \item Use the late-night FM DJ voice
  \item Start with "I'm sorry ..."
  \item Mirror
  \item Silence for at least 4 seconds
  \item Repeat
\end{enumerate}
The intention of a mirror should be 'Please, help me understand'
\begin{itemize}
  \item "I'm sorry, \textit{two copies}?"
\end{itemize}

Mirrors will feel awkward when first trying them, but need to practice. Once mastered, it is very valuable in just about every professional/social setting.

\section{Don't Feel Their Pain, Label It}
Emotion is a powerful tool.

\subsection{Tactical Empathy}
Empathy is the recognition and vocalizaiton of counterpart's perspective. ie making an effort to understand counterpart's feelings and their world, though not to agree.

To practice, while someone is talking to you, imagine you are them talking--being in their position.

\subsection{Labelling}
Labelling is validating someone's emotion by verbalizing it."It looks like you don't want to go to jail". When practicing, always feels like other person would say "Don't tell me how I feel" but no one ever notices it.

Steps to labelling:
\begin{enumerate}
  \item Spot an emotion by picking up cues counterpart drops (words, tone, body language)
  \item "It seems/sounds/looks like ... ?/." Do not use I. If disagreement, say "I didn't say that was what it was. I just said it seems like that"
  \item Silence. Always an urge to specify or lead with a narrower question, but a label's power is to invite counterpart to reveal more by keeping it vague.
\end{enumerate}

To practice: strike up conversation with someone (mailman/daughter) and put a label on their emotions and go silent.

\subsection{Neurtralize the Negative, Reinforce the Positive}
Labelling is a tool not a recipe: it amplifies positives and diffuses negatives. If need to appologize for a mistake, get right at it. Acknowledge the negative situation to diffuse it. Labelling counterpart's feaurs diffuses their power. The earlier those fears are addresed, the earlier genuine feelings of safety and trust can be built.
Likewise, reinforce positive perceptions by using labels.

\subsection{Clear the Road before Advertising the Destination}
Girl scouts example of labels being powerful -- acknowledging the emotion solves everything else:
- "I'm sensing some hesitation with these projects" --> "I want my gift to directly support programming for Girl Scouts and nothing else"
- "It seems that you are really passionate about this gift and want to find the right project reflecting the opportunities and life-chaning experience the Girl Scouts gave you"

\subsection{Do an Accusation Audit}
People afraid to volunteer in front of peers: "In case you're worried about volunteering to role-play with me in front of the class, I want to tell you advance ... it's going to be horrible"
Do not do:
\begin{itemize}
  \item "I don't want to sound harsh"
  \item "I don't want to come off as an asshole"
\end{itemize}
because they deny the negative. Reasons why counterpart will not make a deal more powerful they why they will make a deal: focus on clearing the barriers to agreement.

\subsubsection*{Accusation audit}
Listing every terrible thing counterpart could say about you. At first seems very artifical/self-loathing. Anna contract example: labelled counterpart's fears -- "You feel like we treated you <bad>", "It sounds like you think that we <bad>" -- counterpart then adds nuance to those fears. Then those details gave Anna the power to come to a solution.

\subsection{Get a seat -- and an upgrade -- On a sold-out flight}
Above tools need to be used in unison like the multitude of instruments playing at a band together. At first it may feel very awkward.
\subsubsection*{Ryan example for airline}
\begin{enumerate}
  \item Teller being yelled at by angry couple
  \item "Hi Wendy, I'm Ryan. It seems like they were pretty upset" -- mentioning the couple that left. This is labelling the negative and builds rapport $\rightarrow$ Wendy to elaborate on situation, allowing Ryan to mirror
  \item "Yeah. They missed their connection. We've had many delays due to the weather" met with Ryan's mirror "The weather?" and Wendy explaining the storms.
  \item "It seems like it's been a hectic day" -- futher mirroring and encouraging Wendy to elaborate futher.
  \item "There's been a lot of irate consumers, you know? <example> Many people heading to Austin for the big game". Ryan: "The big game?". Wendy: "<elaborates on the game and flight's booked solid>". Ryan: "Booked solid?". Notice: Ryan using labels and mirrors to build a relationship, using "What's that?" and "I hear you" which allows Wendy to elaborate.
  \item Wendy: "Yeah <stuff>, who knows how many people will make their flights due to rerouting cause of the weather".
  \item Ryan asks with empathy and labelling that he acknowledges her situation: "Well it seems like you've been handling the rough day pretty well. I was also affected by the weather delays and missed by connecting flight. It seems like this flight is likely booked solid, but with what you said, maybe soneome affected by the weather might miss this connection. Is there any possibility a seat will be open?"--Label, tactical empathy, label, then request.
  \item Wendy types, Ryan stays silent.
\end{enumerate}
\subsubsection*{How to practice}
When you see angry customer talking to service person, can practice on them.
\section{Beware 'Yes' -- Master 'No'}
Pushing hard for 'yes' only makes counterpart more defensive. 'No' is very beneficial as it allows both parties to figure out what is getting in the way.
\subsection{"No" starts the negotiation}
People need to feel in control, and allowing your counterpart to say "No" gives them a sense of 'veto' power that they feel preserves their autonomy, making them more open to what you have to say. Do not hope to hear "no" at some point, aim to get them to say it early. Need to internally reframe "no" from a rejection to one of:
\begin{enumerate}
  \item I am not yet ready to agree
  \item You are making me feel uncomfortable
  \item I do not understand
  \item I don't think I can afford it
  \item I want something else
  \item I need more information/talk it over with someone else
\end{enumerate}
Then follow up with solution-based questions:
\begin{enumerate}
  \item What about this doesn't work for you?
  \item What would you need to make it work?
  \item It seems like there's something here that bothers you
\end{enumerate}
\subsection{Persuade in their world}
There are three types of "yes": counterfeit, confirmation, and commitment. Focusing on yourself and backing your opponent to a corner with logical arguments to receive counterfeit 'yes' does not work. Instead want counterpart to feel like they've reached the conclusions themselves (or at least equal part). Every person has two primal urges that both need to be satisfied
\begin{enumerate}
  \item need to feel safe and secure
  \item need to feel in control
\end{enumerate}
\subsection{"No is protection"}
Counterpart saying 'no' makes them feel like they're in the driver's seat. Gunning for 'yes' leads to counterfeit 'yes's. Must get rid of the fear of 'no'. Pitt Police example of supervisor going to fire victim:
\begin{enumerate}
  \item victim: "Do you want the FBI to be embarrassed?" -- loaded question going to get a 'no'
  \item supervisor: "no" -- supervisor feels in control
  \item victim: "what do you want me to do" -- supervisor feels safe
\end{enumerate}
There's many ways to get counterpart to say no: say something you know is not true about them, etc

\subsection{Email magic: How never to be ignored again}
Provoke 'no' with email: \textit{Have you given up on this project?}. Offers counterpart feeling of safety and control by responding and clarifying.

\section{Trigger the Two Words that Immediately Transform Any Negotiation}
Carl Rogers unconditional positive regard (humanistic therapy) opens the door to changing thoughts and behaviours. Need counterpart to open up honestly. Change Stairway Model (BCSM):
\begin{enumerate}
  \item active listening
  \item empathy
  \item rapport
  \item influence
  \item behavioural change
\end{enumerate}
The key two words in any negotiation: "That's right"
\subsection{Create a subtle epiphany}
Find a 'that's right' moment somewhere during the negotiation. It doesn't usually occur in the beginning, and is invisble to them when it occurs but they embrase what you've said.
\subsection{Trigger a "That's Right" with a summary}
\begin{enumerate}
  \item Effective pauses: allow counterpart to keep talking and drain their emotions from the dialogue
  \item Minimal encourages: (\textit{yes, ok, uh-huh, I see}) to note paying attention
  \item Mirroring: repeat back
  \item Labeling: give counterpart's feelings a name and identify (\textit{it all seems so tragically unfair, I can see why you sound so angry})
  \item Paraphrase: repeat in own words to show you understand
  \item Summarize: combine rearticulation of meaning and underlying feelings. Paraphrasing + Labelling gets "that's right"
\end{enumerate}

\subsection{"That's right" is great but if "You're right", nothing changes}
Hearing "that's right" is a winning strategy but "you're right" is a disaster. Linebacker example: \textit{You seem to think it's unmanly to dodge a block? You think it's cowardly to get out of someone's way?}

\subsection{Using "That's right" to me the sale}
Using "That's right" drops counterpart's guard down so they can listen intently. Want to lead to something you know is true: \textit{You seem to tailor specific treatments and medications for each patient}. Also let counterpart explain and then tout how your product can fit in their routine

\subsection{Using "That's right" for career success}
There's always leverage that can be used in an argument.

\subsubsection{Don't Compromise}
Compromising happens as a safe and easy way out. Don't do that, find the novel solutions. No deal is better than a bad deal.

\subsubsection{Deadlines: Make time your ally}
Deadlines are often arbitrary and flexible, also it's beneficial to let your counterpart know your deadline.

\subsubsection{No such thing as fair}
Decisions are based on emotions

\subsubsection{The f-word: why it's so powerful and when to use it}


\section{Bend Their Reality}

\section{Create the Illusion of Control}

\section{Guarantee Execution}

\section{Bargain Hard}

\section{Find the Black Swan}





\end{document}
